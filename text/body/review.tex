\documentclass[tcc]{subfiles}

\begin{document}

\chapter{Literature review}
\label{sec:review}
\epigraph{To fight and conquer in all our battles is not supreme excellence; supreme excellence consists in breaking the enemy's resistance without fighting.}{-Sun Tzu}

The simulation of a gas turbine engine is fundamentally a multidisciplinary matter.
It encompass not only the fields of turbomachinery and compressible flow, 
but also those of dynamical systems and numerical analysis.
This chapter is a literature review on these fundamental fields with the goal of 

Since only the basic equations of compressible flow will be used, they will not be reproduced here. 
The reader should refer to \textcite{Anderson, Shapiro}, 
or simply check the very nice review table available at NASA's website \cite{nasa_isentropic}.

\section{Euler's equation}
\label{sec:euler_equation}
Euler's equation is a consequence of the application of the laws of conservation of angular momentum and energy to a turbomachine. It is the cornerstone of performance calculation for compressor and turbines, and is derived in any textbook on turbomachinery \cite{Lakshminarayana1996, Dixon1998, Schobeiri2004, Hill1991, Logan2003, Baskharone2006}. Due to its importance, this equation will be briefly derived here.

From the conservation of angular momentum, 

\begin{equation}
    \vec{\acs{torque}} = \dot{m} (\vec{r}_{\text{out}}\cross\vec{V}_{\text{out}} - \vec{r}_{\text{in}}\cross\vec{V}_{\text{in}}) 
\end{equation}

Multiplying both sides by the rotational speed ($\omega$)

\begin{equation}
    \acs{power} = \vec{\omega}\cdot\vec{\acs{torque}} 
                = \dot{m} \vec{\omega} \cdot (\vec{r}_{\text{out}}\cross\vec{V}_{\text{out}} - \vec{r_{\text{in}}}\cross\vec{V}_{\text{in}}) 
\end{equation}

Assuming that the system is adiabatic and using the first law of thermodynamics, the specific enthalpy change is given by

\begin{equation}
    \Delta h = \frac{\acs{power}}{\dot{m}} 
             = \vec{\omega} \cdot (\vec{r}_{\text{out}}\cross\vec{V}_{\text{out}} - \vec{r}_{\text{in}}\cross\vec{V}_{\text{in}}) 
\end{equation}

\begin{equation}
    \Delta h = \omega (r_{\text{out}} {V_u}_{\text{out}} - r_{\text{in}} {V_u}_{\text{in}}) 
\end{equation}

\begin{equation}
    \label{eqn:euler}
    \Delta h = U_{\text{out}} {V_u}_{\text{out}} - U_{\text{in}} {V_u}_{\text{in}} 
\end{equation}

\subsection{Euler's equation for a centrifugal compressor}

For the analysis of a centrifugal compressor, 
it is useful to write the absolute tangential velocity at the compressor exit (${V_u}_3$)
as a function of the radial relative velocity (${W_r}_3$),
since the latter is more closely related to the mass flow, 
since $\dot{m} = \rho {W_r}_3 A_3$, where $A_3$ is the ring shaped exit area of the compressor.

From the velocity triangle in figure \todo{add figure}, 
and assuming that the flow exits tangential to the blade surface,
\begin{equation}
    \label{eqn:V_u_3}
    {V_u}_3 = U_3 - {W_r}_3 \tan\beta_3
\end{equation}

%Due to aerodynamic blockage at the tip and flow distortions caused by viscous efects, flow diffusion, clearences and blade aspect ratio,
%${V_u}_3$ is actually smaller than what is predicted by \cref{eqn:V_u_3}.
%This is accounted for by the \acf{impeller_distortion_factor}\index{impeller distortion factor}.
In a real impeller, the flow does not exit perfectly tangential to the blade surface. 
This is accounted by the introduction of a \acf{slip_factor} \cite{Wiesner1967,Aungier1995}.
The corrected equation for the absolute tangential velocity is then

\begin{equation}
    \label{eqn:V_u_3_corrected}
    \boxed{{V_u}_3 = \acs{slip_factor}(U_3 - {W_r}_3 \tan\beta_3)}
\end{equation}

Substituting \cref{eqn:V_u_3_corrected} in \cref{eqn:euler} and dividing both sides by ${U_3^2}$, we have
\begin{equation}
    \acs{load_coef}_c = \acs{slip_factor}(1-\acs{flow_coef}_3\tan\beta_3) - \frac{U_2 {V_u}_2}{U_3^2} 
\end{equation}
where $\phi_3 \triangleq \frac{\dot{m}}{A_3 \rho_3 U_3} = \frac{W_{3r}}{U_3}$ is the flow parameter\index{flow parameter} at the compressor exit
and $\psi_c \triangleq \frac{\Delta h}{U_3^2}$ is the compressor power coefficient.

The second term accounts for pre-whirl\index{pre-whirl} in the flow entering the compressor, e.g due to \ac{IGV}.
In the case of no pre-whirl, ${V_u}_2 = 0$, thus 
\begin{equation}
    \label{eqn:euler}
    \boxed{\acs{load_coef}_c = \acs{slip_factor}(1-\acs{flow_coef}_3\tan\beta_3)}
\end{equation}

\subsection{Euler's equation for an axial turbine}

\begin{figure}
    \sffamily\small 
    \caption{Velocities in an axial turbine stage}
    \label{fig:turbine_euler}
    \hfill\includesvg{fig/turbine_euler}\hfill
    \source{author's figure}
\end{figure}

Applying \cref{eqn:euler} to an axial turbine stage (\cref{fig:turbine_euler}), we have
\begin{equation}
    \Delta_h = U_5V_5-U_4V_5
\end{equation}
The tangential fluid velocities can be expressed in terms of their axial components, i.e.\
\begin{align}
    V_{4u} &= V_{4x} \tan\alpha_4 \\
    V_{5u} &= U_5 + V_{5x} \tan\beta_5
\end{align}
Then,
\begin{equation}
    \Delta h = U_5^2 + U_5 V_{5x}\tan\beta_5 - U_4 V_{4x}\tan\alpha_4
\end{equation}
And finally, making the equation nondimensional by diving both sides by the blade speed at the exit $U_5$, we get
\begin{equation}
    \boxed{\psi_t = 1 + \phi_5\tan\beta_5-\left(\frac{r_4}{r_5}\right)^2\phi_4\tan\alpha_4}
\end{equation}
where $\phi_4 = \frac{V_{4x}}{U_5}$ and $\phi_5 = \frac{V_{5x}}{U_5}$ are the flow parameters and
$\psi_t=\frac{\Delta h}{U_5^2}$ is the power coefficient for the turbine.

\section{Loss models for centrifugal compressors}
\label{sec:compressor_losses}
Euler's equation is only valid for isentropic flow. 
For a real flow, corrections must be introduced to account for the generation of entropy in many parts of the compression process. 
These so called ``losses'' are modelled in literature as enthalpy corrections 
for the Euler equation considering both the \emph{actual} process and the \emph{isentropic} process 
that would generate the same increase in pressure as the real process.
Each of these losses can be categorized as either \emph{internal} or \emph{parasitic}.
Internal losses are those that happen inside the fluid main flow, and result in a decrease in the resulting pressure ratio.
Parasitic losses are the ones who happen outside the fluid main flow, 
and result in a increase in the temperature ratio through the compressor \cite{Galvas1973}.
Using these definitions, the actual and isentropic enthalpy changes through the compressor can be calculate as follows

\begin{align}
    \Delta h_{\text{actual}} &= \Delta h_{\text{euler}} + \sum_{\substack{
                                                            \text{parasitic} \\ 
                                                            \text{losses}
                                                        }}
                                                          \Delta h \\
    \Delta h_{\text{isen}}  &= \Delta h_{\text{euler}} - \sum_{\substack{
                                                            \text{internal} \\ 
                                                            \text{losses}
                                                        }}
                                                          \Delta h
\end{align}

The temperature and pressure ratios are given by
\begin{align}
    \frac{T_{03}}{T_{02}} &= 1 + \frac{\Delta h_{\text{actual}}}{c_p T_{02}} \\
    \frac{P_{03}}{P_{02}} &= \left( 1 + \frac{\Delta h_{\text{isen}}}{c_p T_{02}}\right)^{\frac{\gamma}{\gamma-1}}
\end{align}

The isentropic and polytropic efficiencies can then be readly calculated as
\begin{align}
    \eta_t &= \frac{\Delta h_{\text{isen}}}{\Delta h_{\text{actual}}}\\
    \eta_p &= \frac{\gamma-1}{\gamma} \left(\frac{\log\frac{P_{02}}{P_{01}}}
                                                 {\log\frac{T_{02}}{T_{01}}}
                                     \right)
\end{align}

\textcite{Aungier1995} criticises this calculation because the losses should be regarded as changes in entropy, and not in enthalphy. Despite this shortcomming, the enthalphy approach is widely used.

The losses generally accounted for are listed in \cref{tbl:compressor_loss_mechanisms}. According to \textcite{Gravdahl2004,Gravdahl1999,Ferguson1963,Watson1982}, the most important losses for stability considerations are the incidence losses and the skin friction losses. The former is destabilizing because it introduces the positive slope of the left hand side of the compressor map. The fluid friction losses are stabilizing because they increase with mass flow, therefore contributing to the decrease in the derivative of the iso-speed lines.
\begin{table}
\caption{Loss mechanisms for a centrifugal compressor}
\label{tbl:compressor_loss_mechanisms}
\hrule
\begin{multicols}{2}
\begin{compactitem}
    \item[] \textbf{Internal}
    \item Incidence loss (in the impeller and diffuser)
    \item Blade loading loss
    \item Skin friction loss
    \item Clearance loss
    \item Mixing loss
    \item Vaneless diffuser loss
    \columnbreak
    \item[] \textbf{Parasitic}
    \item Disc friction loss
    \item Recirculation loss
    \item Leakage loss
\end{compactitem}
\end{multicols}
\hrule
\source{\cite{Oh1997}}
\end{table}

A detailed discussion of each loss mechanism is beyond the scope of this work; 
the interested reader should refer to \textcite{Cumpsty2004}. 

Semi-empirical models are available to predict each individual loss based on compressor geometry and flow conditions. 
In particular, \textcite{Oh1997} did a review of various loss models for each loss mechanism and selected an optimal set.

\section{Loss models for axial turbines}

Most of the empirical loss correlations for axial turbines in use today stem from the seminal work by \textcite{Ainley1951}. This work identified the main loss mechanisms for an axial turbine stage as 
\begin{description}
    \item[profile loss] due to skin friction and flow separation under high incidence angles of the blades
    \item[annulus loss] due to boundary layer growth on the inner and outer walls of the turbine annulus
    \item[secondary flow loss] due to the tri-dimensional nature of the flow
    \item[tip clearence loss] due to the vortex formed at the tip of the turbine blades
\end{description}

Several improvements were made to Ainley and Mathieson's model by \textcite{Mukhtarov196, Dunham1970, Kacker1982, Moustapha1990, Benner1995} among others\cite{Persson2015}. All of these studies were based on empirical data available from previous turbine rig testing, and thus are valid for turbines with similar geometries.
These models provide the loss values as either a pressure loss-coefficient
\begin{equation}
    Y = \frac{P_{04}-P_{05}}{P_{05}-P_5}
\end{equation}
or a kinectic energy loss coefficient
\begin{equation}
    \phi^2 = \left(\frac{\text{actual gas exit velocity}}
                        {\text{ideal gas exit velocity}}\right)^2
\end{equation}

\section{Deviation models for axial turbines}

The flow exiting a turbine blade row does not follow the blade camber line exactly. 
This is analogous to the slip observed in centrifugal compressors, but needs more elaborate correlations, 
thus deserving a section of its own. 
The simplest correlation available is known as Carter's rule \cite{Mattingly1996}, and states that
\begin{equation}
    \label{eqn:cartersrule}
    \delta = \frac{\beta_4-\beta_5}{4\sqrt{\sigma}}
\end{equation}
where $\delta$ is the angular deviation between the exit flow and the exit camber angle,
$\beta_4$ and $\beta_5$ are the inlet and exit airfoil camber angles,
and $\sigma=\frac{\text{chord}}{\text{blade separation}}$ is the solidity.

Another model based on inlet and exit Mach numbers is provided by \textcite{Ainley1951}, 
and a more elaborate model involving also blade thickness was proposed by \textcite{Islam1999}, where
\begin{equation}
    \delta = \frac{\frac{\rho_5 V_{x4}}{\rho_5 V_{x5}} \sigma^{-1.1} (\alpha'_4+\beta_5)^{2.25}}
                        {\xi^{1.45} \left(t_{\max}/c\right) \left(22+0.22\beta_4^{1.64}\right)}
\end{equation}
where $\xi$ is the blade geometric incidence (stagger angle), 
$t_{\max}/c$ is the blade maximum thickness-to-chord ratio 
and all angles are given in degrees.

\section{\Acl{GTE} dynamics}

A simple engine dynamical model is due to \textcite{Fink1992}. 
This model was validated experimentally by \textcite{Gravdahl2004}.
It is a system of three \acp{ODE}.

\begin{align}
    \dot{\rho}_{04} &= \frac{1}{\acs{volume}}(\dot{m}_c - \dot{m}_t) \\
    \ddot{m}_c &= \frac{A_1}{L_c}(P_{03}-P_{04}) \\
    \dot{\omega} &= \frac{1}{J} (\tau_t-\tau_c)
\end{align}

These equations represent, respectively, 
the conservation of mass and momentum in the burner
and the conservation of angular momentum in the spool.

%Available programs
There are many readily available computer programs that implement this kind of simulation. 
From these, the most famous is probably the commercial program GasTurb \cite{GasTurb}, 
\index{software} \index{GasTurb}
followed by the \gls{GSP} \cite{Visser2000}.
A comparison of both is presented in \textcite{GasTurbvsGSP}.
In particular, GasTurb has already been used to simulate model gas turbines 
\cite{gao2011modelling}.
An open-source alternative is the \gls{T-MATS}, from NASA \cite{T-MATS}.
\gls{T-MATS} provides a library of turbo machinery blocks for use in Simulink. 
Each of these blocks is actually a wrapper for a function written in C 
that simulates the component's behaviour.

The downside of all these programs is that they rely on component map data provided by the user 
and do not consider the burner pressure and mass flow as state parameters---only the spool speed.
The generation of the maps is a substantial effort that must be carried independently, 
either by scaling maps from similar designs, obtaining the maps experimentally 
or using semi-empirical methods as described in this work. 
On the other hand, disregarding the dynamical behaviour of the combustion chamber makes the model unable to predict surge on its own. The surge can still be included through a surge line in the compressor map but,
depending on how the map was obtained, this line may not be accurate because the surge depends on the components downstream from the compressor, as discussed in \cref{sec:review:surge}.

\section{Nondimensional parameters}
\label{sec:nondimensional}

Using the Buckinham $\Pi$ Theorem \cite{Buckingham1914},
the parameters which govern engine and component performance can be grouped into new, nondimensional parameters. 
This has the advantage that the engine behaviour becomes invariant by phisical size, and so the behaviour of a small model engine resembles that of a large commercial gas turbine. 
It is specially beneficial for choosing initial guesses for numerical solvers.

\textcite[chapter 4]{walsh2004gas} provide some very comprehensive tables 
for the nondimensional groups used for both individual components and complete engines.
Componentwise, the relevant parameters are the \acl{gam}, the \acl{MFP} at inlet and exit, t
he \acl{M0rotor}, the stagnation temperature ratio, and the stagnation pressure ratio. 
These quantities are defined respectively as
\begin{align}
    \gamma &\triangleq \frac{c_p}{c_v} \\
    \acs{MFP}_i &\triangleq \MFP{i} = \MFPalt{i} \label{eqn:MFPi}\\
    \acs{MFP}_e &\triangleq \MFP{e} \\
    \acs{M0rotor} &\triangleq \frac{U}{\sqrt{\gamma R T_{0i}}} \\
    \text{temperature ratio}&\triangleq\tr{0e}{0i} \\
    \text{pressure ratio}&\triangleq\pr{0e}{0i}
\end{align}
where \ldots
The temperature ratio can be equivalently replaced by the adiabatic or polytropic efficiency,
see \cref{sec:compressor_losses}.

In particular, the \acl{MFP} is a function of the Mach number and the specific heat ratio,
 i.e.\
\begin{equation}
    \label{eqn:mfp2mach}
    \acs{MFP}(M,\gamma) = M \left( 1 + \frac{\gamma-1}{2}M^2\right)^{-\frac{\gamma+1}{2(\gamma-1)}}
\end{equation}
While there is no closed form inverse for this function, it is invertible and the inverse can be easily computed numerically, e.g.\ by a newton scheme \cite{Der1974}.

Although the Reynolds number also has an impact on component performance, 
its effects are of second order\cite{walsh2004gas}. 
For the purpose of this work, it is only a parameter of secondary importance 
that can be eventually used in the computation of loss models.

For the entire engine, the relevant nondimensional groups are the stationwise
\acl{MFP}, nondimensional temperature and nondimensional pressure;
the reference \acl{M0rotor} and the nondimensional fuel mass flow. 
The stationwise \acl{MFP} is defined in the same way as in \cref{eqn:MFPi}, 
i.e.\ for station $n$, $\acs{MFP}_n = \MFP{n}$. 
The nondimensional pressures and temperatures are the local stagnation property normalize by the ambient value. The reference \acl{M0rotor} is taken as the compressor's 
and the nondimensional fuel flow parameter is defined as 
$\frac{\dot{m}_f \sqrt{RT_{01}}}{A_1 P_{01} \sqrt{\gamma}}$.

\section{Engine stability: surge}
\label{sec:review:surge}

There are two main threats to the stability of a \ac{GTE}: surge and flame extinction.
Flame extinction happens when the fuel to air ratio and combustion chamber temperature 
exit a certain stability envelope \cite{Mattingly1996}, 
so it does not need to be accounted for directly in an engine model, and thus will not be covered here.
Surge is a dynamical instability in a compressor system caused by its interaction with the components downstream \cite{Sparks1983}. 
It is characterized by periodic backflow accompanied by vibration, a marked dip in thrust and rotation speed, and a characteristic sound, 
which can be a series of bangs for engines with large internal volume or a low frequency hum for smaller engines.
In extreme cases, it can cause damage to the engine. 
Normally the system can not recover from surge on its own, 
and either a reduction in fuel flow or the opening of bleed valves is required to return the engine to its steady state operation.
\cite{Willems1999}.

The surge phenomenon arises naturally from the dynamical model of a compressor system if pressure transients are taken into account \cite{Greitzer1976}, 
and has good agreement with experimental data \cite{Greitzer1976_2,Gravdahl2004}.
It is nevertheless good to have a criterion for the onset of surge, if only to draw the surge line on the compressor map.
A simple model for surge with all the fundamental elements is due to \textcite{Fink1988}. This model includes a compressor, a plenum of known volume, some plumbing of known length and an adjustable throttle valve. 
In an actual engine, the plenum and plumbing are analogous to the combustion chamber, while the throttle can be considered as a simplified turbine-nozzle assembly. 
The stability criterion derived from this model by linear control theory is

\begin{align}
    C' &< \frac{1}{B^2 T'} \label{eqn:surge} \quad\text{and}\\
    C' &< T'
\end{align}
where $C' = \frac{\partial \pr{03}{02}}{\partial \dot{m}_c}$ is the derivative of the compressor characteristic, $T' = \frac{\partial \pr{05}{04}}{\partial \dot{m}_t}$ is the derivative of the turbine characteristic and 
\begin{equation}
    B = \frac{U}{2a}\sqrt{\frac{\acs{volume}}{A_c L_c}}
\end{equation}
where $A_c$ and $L_c$ are the area and the length of the pipe downstream from the compressor.

This model provides a good physical explanation of the mechanism of surge. 
First, one should notice that, by the first stability criterion, surge can only occur when the compressor characteristic has a positive derivative, 
i.e.\ when an increase in mass flow is accompanied by an increase in the pressure ratio supplied by the compressor. 
If this increase in pressure ratio does not lead to an increased mass flow through the throttle, the increase in compressor mass flow ration will increase the plenum pressure above that which the compressor can supply, and this will result in backflow. In the negative part of the compressor characteristic, it will be able again to supply the demanded pressure, now made smaller due to backflow. Since this condition will lead to an ever decrease in the plenum pressure, the flow will eventually invert its direction again, and this results in the periodic behaviour of surge.

Analysing the extreme values of the parameter $B$ is helpful in developing an intuitive grasp for the phenomenon.
For large values of $B$, that is, large burner volumes, the surge limit approaches the point where the compressor characteristic has zero derivative. Since the plenum is very big, any change in the mass flow of the throttle will make a very small difference in the plenum pressure, and so any simultaneous increase in compressor mass flow and pressure ratio will lead to instability. This is the conservative limit for the surge line.
Inversely, for small values of $B$, that is, small burner volumes, any increase in pressure will lead to a fast increase in mass flow through the throttle, which will instantaneously reduce the pressure, thus having an stabilizing effect. 
In this case, the two criteria for surge collapse into one.

Another result from this model is that while surge is often related to compressor stall, this is not necessarily the case.
The surge line for a compressor system can be moved be changing components downstream from it, and operation of compressors beyond the surge line, i.e.\ with flow angles that would otherwise cause surge, using control systems based on bleed valves is widely reported in the literature \cite[e.g.][]{Liaw2004}, and that would not be possible if surge was simply an aerodynamical effect.

\section{Nonlinear systems}
\label{sec:review:numeric}

As will be shown in \cref{sec:methods}, the modelling of a gas turbine engine leads naturally to nonlinear systems of the type 
\begin{equation}
    \label{eqn:nonlinear}
    f(x) = 0, \quad f:\mathbb{R}^n \rightarrow \mathbb{R}^n
\end{equation}
for component maps and static engine operating points, and to \acl{DAE} systems of the type
\begin{equation}
    \label{eqn:dae}
    f(t, x, \dot{x}) = 0, \quad f:\mathbb{R}^{2n+1} \rightarrow \mathbb{R}^n
\end{equation}
for the dynamical engine model. 
This particular formulation is know in the literature as a \ac{DAE} system \cite{Brenan1995}.
The surge model, on the other hand, is best expressed as a partial \acl{DAE} system of the form
\begin{equation}
    f(x, \nabla x) = 0, \quad f:\mathbb{R}^{n^2+n} \rightarrow \mathbb{R}^n
\end{equation}
where $\nabla x$ is the Jacobian of $x$. 
However, the literature on this type of system is sparse at best and no widely used software package could be found.
This section will therefore focus itself on the problems which can be expressed in the form of \cref{eqn:nonlinear,eqn:dae}.

\textcite{walsh2004gas} suggests that one way of solving theses systems is solving for one variable at a time in an approach known as successive loop closure or serial nested loop. 
For example, in the case of an engine's static model, \textcite{walsh2004gas} proposes that, starting with a fixed shaft speed, one first guesses the operating point on the compressor map and the turbine inlet temperature $T_{04}$. Then, $T_04$ is adjusted until the mass flow predicted by the turbine map matches the guess for the turbine map. The nozzle performance is then evaluated and the position on the compressor map is changed to match the nozzle mass flow ratio. This procedure is repeated until convergence of both parameters is obtained.

This method is severely flawed. As remarked by \textcite{walsh2004gas}, it becomes very inefficient when the number of guesses is more than approximately five. Furthermore, this technique is not only hard to set up and dependent on quite a lot of intuition acquired through experience, but it is also very prone to diverge (sometimes wildly so) if the updated values for the variables are not chosen just right. A more robust approach is to use proven solvers for non-linear systems, referred by \textcite{walsh2004gas} as the ``matrix'' solution method. 
This method consists on using gradient based algorithms to solve the system simultaneously for all variables.

There are two main approaches to solving non-linear systems numerically in wide use today. 
One is to use classical root finding algorithms based on optimization theory 
and the other is to use path following methods, such as
\ac{DAE} system solvers and homotopy methods \cite{Allgower1997,Rabier2002,hompack,sundials}. 

Classical methods for finding roots of nonlinear systems are based on optimization theory. 
They are best represented by the software package \textsc{minpack} \cite{minpack}, which includes implementations of Powell's hybrid method and Levenberg-Marquardt's method.

It should be noted that every system expressed as a nonlinear system can also be represented as a \ac{DAE}. 
This is most useful for under determined systems where not only one solution is sought 
but a uni-dimensionally parameterized continuum of solutions, such as an operating line, 
choke line, surge line, etc.

\end{document}
