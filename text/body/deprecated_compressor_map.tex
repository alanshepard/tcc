The procedure for calculating the compressor map is summarized in \Cref{alg:comp_map}. 
This is a iterative process, and initial guesses must be made for both 
the mass flow parameter at the impeller exit ($\acs{MFP}_2$) and the temperature ratio 
($\tfrac{T_{02}}{T_{01}}$). 
It is important to keep the values for $\acs{MFP}_2$ bellow the choke value during all iterations,
otherwise the computation of $M_2$ will fail. 
Therefore, it is convenient to guess that the compressor exit is choked because this will lead 
to approaching the actual value of $\acs{MFP}_2$ from above.
Even if the exit is far from choked, the method converges very rapidly in our experience.

\begin{algorithm}
    \caption{Compressor map}
    \label{alg:comp_map}

    \KwIn{\textsc{Inputs:} \acs{MFP}, \acs{M0rotor}, \acs{gam}}
    
    guess that the outlet is choked, i.e.\ 
        $\acs{MFP}_2 = \left( \frac{\gamma+1}{2}\right)^{-\frac{\gamma+1}{2(\gamma-1)}} $\;
    guess that the compressor is isentropic, and take
     $\frac{T_{02}}{T_{01}} = \left(\frac{A_2}{A_1}\frac{\acs{MFP}_2}{\acs{MFP}_1}\right)^{-\frac{2(\gamma-1)}{\gamma+1}} $\;
    \Repeat{$\acs{MFP}_2$ converges} {
        calculate $M_2$ by solving \Cref{eqn:mfp2mach}\;
        calculate $\phi_2$ using \Cref{eqn:phi2_dimensionless}\;
        calculate $\psi_\text{euler}$ from Euler's~equation~(\ref{eqn:euler})\;
        calculate parasitic and 
        calculate $\frac{T_{02}}{T_{01}}$ and $\frac{P_{02}}{P_{01}}$ 
            using \Cref{eqn:psi2Tratio,eqn:psi2Pratio} \;
        calculate $\acs{MFP}_2$ from \Cref{eqn:mfp2mfp2}\;
}

    \KwOut{$\frac{T_{02}}{T_{01}}$, $\frac{P_{02}}{P_{01}}$}
\end{algorithm}

