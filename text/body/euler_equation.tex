\documentclass[tcc]{subfiles}
\begin{document}


\chapter{Compressor model}

Assumptions: ideal gas

\section{Work done in a compressor: Euler's equation}
\label{sec:euler_equation}

From the conservation of angular momentum, 

\begin{equation}
    \vec{\acs{torque}} = \dot{m} (\vec{r_2}\cross\vec{V_2} - \vec{r_1}\cross\vec{V_1}) 
\end{equation}

Multiplying both sides by the rotational speed ($\omega$)

\begin{equation}
    \acs{power} = \vec{\omega}\cdot\vec{\acs{torque}} 
                = \dot{m} \vec{\omega} \cdot (\vec{r_2}\cross\vec{V_2} - \vec{r_1}\cross\vec{V_1}) 
\end{equation}

Assuming that the system is adiabatic and using the first law of thermodynamics the specific enthalpy change is given by,

\begin{equation}
    \Delta h = \frac{\acs{power}}{\dot{m}} 
             = \vec{\omega} \cdot (\vec{r_2}\cross\vec{V_2} - \vec{r_1}\cross\vec{V_1}) 
\end{equation}

\begin{equation}
    \Delta h = \omega (r_2 {V_u}_2 - r_1 {V_u}_1) 
\end{equation}

\begin{equation}
    \label{eqn:euler}
    \Delta h = U_2 {V_u}_2 - U_1 {V_u}_1 
\end{equation}

It is useful to write the absolute tangential velocity at the compressor exit (${V_u}_2$)
as a function of the radial relative velocity (${W_r}_2$),
since the latter is more closely related to the mass flow 
(i.e. $\dot{m} = \rho {W_r}_2 A_2$, where $A_2$ is the ring shaped exit area of the compressor)

From the velocity triangle in figure X, assuming that the flow exits tangential to the blade surface,
\begin{equation}
    \label{eqn:V_u_2}
    {V_u}_2 = U_2 - {W_r}_2 \tan\beta_2
\end{equation}

Due to aerodynamic blockage at the tip and flow distortions caused by viscous efects, flow diffusion, clearences and blade aspect ratio,
${V_u}_2$ is actually smaller than what is predicted by \cref{eqn:V_u_2}.
This is accounted for by the \acf{impeller_distortion_factor}\index{impeller distortion factor}.
Furthermore, the flow does not exit perfectly tangential to the blade surface, and this is accounted by the \acf{slip_factor} \cite{Wiesner1967,Aungier1995}.
The corrected equation for the absolute tangential velocity is then

\begin{equation}
    \label{eqn:V_u_2_corrected}
    {V_u}_2 = \acs{slip_factor}(U_2 - \acs{impeller_distortion_factor}{W_r}_2 \tan\beta_2)
\end{equation}

Substituting \cref{eqn:V_u_2_corrected} in \cref{eqn:euler} and dividing both sides by ${U_2^2}$, we have

\begin{equation}
    \acs{load_coef} = \acs{slip_factor}(1-\acs{impeller_distortion_factor}\acs{flow_coef}_2\tan\beta_2) - \frac{U_1 {V_u}_1}{U_2^2} 
\end{equation}

where $\phi_2 \triangleq \frac{\dot{m}}{A_2 \rho_2 U_2} = \frac{W_{2r}}{U_2}$ is the flow parameter\index{flow parameter} at the compressor exit
and $\psi \triangleq \frac{\Delta h}{U_2^2}$

The second term accounts for pre-whirl\index{pre-whirl} in the flow entering the compressor, e.g due to \ac{IGV}.
In the case of no pre-whirl, ${V_u}_1 = 0$, thus 

\begin{equation}
    \label{eqn:euler}
    \acs{load_coef} = \acs{slip_factor}(1-\acs{impeller_distortion_factor}\acs{flow_coef}_2\tan\beta_2)  
\end{equation}

\section{Losses}
\label{sec:compressor_losses}
There is generation of entropy in many parts of the compression process. 
These so called ``losses'' are modelled in literature as enthalpy corrections 
for the Euler equation considering both the \emph{actual} process and the \emph{isentropic} process 
that would generate the same increase in pressure as the real process \cite{Galvas1973}. 

\begin{align}
    \Delta h_{\text{actual}} &= \Delta h_{\text{euler}} + \sum_{\substack{
                                                            \text{parasitic} \\ 
                                                            \text{losses}
                                                        }}
                                                          \Delta h \\
    \Delta h_{\text{isen}}  &= \Delta h_{\text{euler}} - \sum_{\substack{
                                                            \text{internal} \\ 
                                                            \text{losses}
                                                        }}
                                                          \Delta h
\end{align}

The isentropic and polytropic efficiencies can then be readly calculated as
\begin{align}
    \eta_t &= \frac{\Delta h_{\text{isen}}}{\Delta h_{\text{actual}}}\\
    \eta_p &= \frac{\gamma-1}{\gamma} \left(\frac{\log\frac{P_{02}}{P_{01}}}
                                                 {\log\frac{T_{02}}{T_{01}}}
                                     \right)
\end{align}

The losses generally accounted for are \cite{Oh1997}:
\begin{compactitem}
    \item[] \textbf{Internal}
    \item Incidence loss (in the impeller and diffuser)
    \item Blade loading loss
    \item Skin friction loss
    \item Clearance loss
    \item Mixing loss
    \item Vaneless diffuser loss
    \item[] \textbf{Parasitic}
    \item Disc friction loss
    \item Recirculation loss
    \item Leakage loss
\end{compactitem}

A detailed discussion of each loss mechanism is beyond the scope of this work; 
the interested reader should refer to \textcite{Cumpsty2004}. 

Semi-empirical models were used to predict each individual loss 
based on compressor geometry and flow condition. 
The incidence loss model used assumes that all kinectic energy from the velocity components
normal to the inducer and diffuser blades is lost \textcite{Stanitz1953}.
All other loss models used were suggested by the comparative study of \textcite{Oh1997}, 
and will not be discussed here.

\section{Dimensionless compressor analysis}
\label{sec:dimensionless_compressor}

From the Buckinham $\Pi$ Theorem \cite{Buckingham1914}, 
it can be shown that the compressor behaviour is a function of only three parameters, 
namely the \acl{MFP}, the \acl{M0rotor}, the \acl{gam} and the \acl{Re}. 
For a centrifugal compressor, this quantities are defined respectively as follows

\begin{align}
    \acs{MFP} &\triangleq \MFP{1} = \MFPalt{1} \\
    \acs{M0rotor} &\triangleq \frac{U_2}{\sqrt{\gamma R T_{01}}} \\
    \gamma &\triangleq \frac{c_p}{c_v} \\
    \acs{Re} &
\end{align}

In particular, the \acl{MFP} is a function of the Mach number and the specific heat ratio,
 i.e.\
\begin{equation}
    \label{eqn:mfp2mach}
    \acs{MFP}(M,\gamma) = M \left( 1 + \frac{\gamma-1}{2}M^2\right)^{-\frac{\gamma+1}{2(\gamma-1)}}
\end{equation}
While there is no closed form inverse for this function, it is invertible and the inverse can be easily computed nummerically, e.g.\ by a newton scheme \cite{Der1974}.

According to \textcite{walsh2004gas},
 the influence of the \acl{Re} in the system behaviour is only of second order.

To write Euler's equation \cref{eqn:euler} in terms of these parameters, suffices to rewrite $\phi_2$ and $\psi$ as a function of them. Thus
\begin{multline}
    \label{eqn:phi2_dimensionless}
    \phi_2 = \frac{W_{2r}}{U_2} 
           = \frac{\frac{\dot{m}}{\rho_2 A2}}{\acs{M0rotor} a_{01}}
           = \frac{\MFPalt{2}}{\acs{M0rotor}} \frac{a_{02}}{a_{01}} \stagdensratio{2} \\ 
           = \frac{\acs{MFP}_2}{\acs{M0rotor}} \sqrt{\frac{T_{02}}{T_{01}}} \stagdensratio{2}
\end{multline}
The value of $M_2$ can be obtained from $\acs{MFP}_2$ using \cref{eqn:mfp2mach}, while the latter is given by
\begin{equation}
    \label{eqn:mfp2mfp2}
    \acs{MFP}_2 \triangleq \MFP{2} = \acs{MFP} \frac{A_1}{A_2} \frac{P_{01}}{P_{02}} \sqrt{\frac{T_{02}}{T_{01}}}
\end{equation}
Notice that \cref{eqn:mfp2mfp2} makes no assumption of the process 1--2 being isentropic. $\acs{MFP}_2$ can be used along with \acs{MFP} to check for choking.

The load coefficient is given by
\begin{equation}
    \acs{load_coef}_{2 \text{euler}} \triangleq \frac{\Delta h_{\text{euler}}}{U_2^2}
                      = \frac{c_p(T_{02}-T_{01})}{U_2^2}
                      = \frac{\frac{c_p(T_{02}-T_{01})}{\gamma R T_{01}}}
                                    {\acs{M0rotor}^2}
                      = \frac{\frac{T_{02}}{T_{01}}-1}
                                  {\acs{M0rotor}^2}
                        \left(\frac{1}{\gamma-1}\right)
\end{equation}

At this point the loss models discussed in \cref{sec:compressor_losses} are introduced, 
and the values of $\acs{load_coef}_{2\text{actual}}$ and $\acs{load_coef}_{2\text{isen}}$ are obtained.
The temperatures and pressure ratio can then be readly calculate from
\begin{align}
    \label{eqn:psi2Tratio}
    \frac{T_{02}}{T_{01}} &= (\gamma - 1)\psi_{2\text{actual}} \acs{M0rotor}^2 + 1 \\
    \label{eqn:psi2Pratio}
    \frac{P_{02}}{P_{01}} &= \left[(\gamma - 1)\psi_{2\text{isen}} \acs{M0rotor}^2 + 1\right]^\frac{\gamma}{\gamma-1}
\end{align}

The procedure for calculating the compressor map is summarized in \cref{alg:comp_map}. 
This is a iterative process, and initial guesses must be made for both 
the mass flow parameter at the impeller exit ($\acs{MFP}_2$) and the temperature ratio 
($\tfrac{T_{02}}{T_{01}}$). 
It is important to keep the values for $\acs{MFP}_2$ bellow the choke value during all iterations,
otherwise the computation of $M_2$ will fail. 
Therefore, it is convenient to guess that the compressor exit is choked because this will lead 
to approaching the actual value of $\acs{MFP}_2$ from above.
Even if the exit is far from choked, the method converges very rapidly in our experience.

\begin{algorithm}
    \caption{Compressor map}
    \label{alg:comp_map}

    \KwIn{\textsc{Inputs:} \acs{MFP}, \acs{M0rotor}, \acs{gam}}
    
    guess that the outlet is choked, i.e.\ 
        $\acs{MFP}_2 = \left( \frac{\gamma+1}{2}\right)^{-\frac{\gamma+1}{2(\gamma-1)}} $\;
    guess that the compressor is isentropic, and take
     $\frac{T_{02}}{T_{01}} = \left(\frac{A_2}{A_1}\frac{\acs{MFP}_2}{\acs{MFP}_1}\right)^{-\frac{2(\gamma-1)}{\gamma+1}} $\;
    \Repeat{$\acs{MFP}_2$ converges} {
        calculate $M_2$ by solving \cref{eqn:mfp2mach}\;
        calculate $\phi_2$ using \cref{eqn:phi2_dimensionless}\;
        calculate $\psi_\text{euler}$ from Euler's~equation~(\ref{eqn:euler})\;
        calculate parasitic and 
        calculate $\frac{T_{02}}{T_{01}}$ and $\frac{P_{02}}{P_{01}}$ 
            using \cref{eqn:psi2Tratio,eqn:psi2Pratio} \;
        calculate $\acs{MFP}_2$ from \cref{eqn:mfp2mfp2}\;
}

    \KwOut{$\frac{T_{02}}{T_{01}}$, $\frac{P_{02}}{P_{01}}$}
\end{algorithm}

\end{document}
