\documentclass[tcc]{subfiles}
\begin{document}

\subsection{Dimensionless compressor analysis}

\begin{gather}
    \acs{MFP} \triangleq \MFP{1} = \MFPalt{1} \\
    \acs{M0rotor} \triangleq \frac{U_2}{\sqrt{\gamma R T_{01}}} \\
    \gamma
\end{gather}

\begin{equation}
    \phi_2 \triangleq \frac{W_{2r}}{U_2} 
           = \frac{\frac{\dot{m}}{\rho_2 A2}}{\acs{M0rotor} a_{01}}
           = \frac{\MFPalt{2}}{\acs{M0rotor}} \frac{a_{02}}{a_{01}} \stagdensratio{2}
           = \frac{\acs{MFP}_2}{\acs{M0rotor}} \sqrt{\frac{T_{02}}{T_{01}}} \stagdensratio{2}
\end{equation}


\subsection{Euler's equation}

From the conservation of angular momentum, 

\begin{equation}
    \vec{\acs{torque}} = \dot{m} (\vec{r_2}\cross\vec{V_2} - \vec{r_1}\cross\vec{V_1}) 
\end{equation}

Multiplying both sides by the rotational speed ($\omega$)

\begin{equation}
    \acs{power} = \vec{\omega}\cdot\vec{\acs{torque}} 
                = \dot{m} \vec{\omega} \cdot (\vec{r_2}\cross\vec{V_2} - \vec{r_1}\cross\vec{V_1}) 
\end{equation}

Assuming that the system is adiabatic and using the first law of thermodynamics,

\begin{equation}
    \Delta h = \frac{\acs{power}}{\dot{m}} 
             = \vec{\omega} \cdot (\vec{r_2}\cross\vec{V_2} - \vec{r_1}\cross\vec{V_1}) 
\end{equation}

\begin{equation}
    \Delta h = \omega (r_2 {V_u}_2 - r_1 {V_u}_1) 
\end{equation}

\begin{equation}
    \label{eqn:euler}
    \Delta h = U_2 {V_u}_2 - U_1 {V_u}_1 
\end{equation}

It is useful to write the absolute tangential velocity at the compressor exit (${V_u}_2$)
as a function of the radial relative velocity (${W_r}_2$),
since the latter is more closely related to the mass flow 
(i.e. $\dot{m} = \rho {W_r}_2 A_2$, where $A_2$ is the ring shaped exit area of the compressor)


From the velocity triangle in figure X, assuming that the flow exits tangential to the blade surface,
\begin{equation}
    \label{eqn:V_u_2}
    {V_u}_2 = U_2 - {W_r}_2 \tan\beta_2
\end{equation}

Due to aerodynamic blockage at the tip and flow distortions caused by viscous efects, flow diffusion, clearences and blade aspect ratio,
${V_u}_2$ is actually smaller than what is predicted by \cref{eqn:V_u_2}.
This is accounted for by the \acf{impeller_distortion_factor}.
Furthermore, the flow does not exit perfectly tangential to the blade surface, and this is accounted by the \acf{slip_factor} \cite{Wiesner1967,Aungier1995}.
The corrected equation for the absolute tangential velocity is then

\begin{equation}
    \label{eqn:V_u_2_corrected}
    {V_u}_2 = \acs{slip_factor}(U_2 - \acs{impeller_distortion_factor}{W_r}_2 \tan\beta_2)
\end{equation}

Substituting \cref{eqn:V_u_2_corrected} in \cref{eqn:euler} and dividing both sides by ${U_2^2}$, we have

\begin{equation}
    \acs{load_coef} = \acs{slip_factor}(1-\acs{impeller_distortion_factor}\acs{flow_coef}_2\tan\beta_2) - \frac{U_1 {V_u}_1}{U_2^2} 
\end{equation}

The second term accounts for pre-whirl in the flow entering the compressor, e.g due to \ac{IGV}.
In the case of no pre-whirl, ${V_u}_1 = 0$, thus 

\begin{equation}
    \acs{load_coef} = \acs{slip_factor}(1-\acs{impeller_distortion_factor}\acs{flow_coef}_2\tan\beta_2)  
\end{equation}

\end{document}
