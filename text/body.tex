\documentclass[tcc]{subfiles}

\begin{document}
\textual

\subfile{text/body/introduction}

\chapter{Literature review}
\epigraph{To fight and conquer in all our battles is not supreme excellence; supreme excellence consists in breaking the enemy's resistance without fighting.}{-Sun Tzu}

The simulation of a gas turbine engine is fundamentally a multidisciplinary matter.
It encompass not only the fields of turbomachinery and compressible flow, 
but also those of dynamical systems and numerical analysis.

Since only the basic equations of compressible flow will be used, they will not be reproced here. The reader should refer to \textcite{Anderson, Shapiro}, or simply check the very nice review table available at NASA's website \cite{nasa_isentropic}.

\section{Euler's equation}
\label{sec:euler_equation}
Euler's equation is a consequence of the application of the laws of conservation of angular momentum and energy to a turbomachine. It is the cornerstone of performance calculation for compressor and turbines, and is derived in any textbook on turbomachinery \cite{Lakshminarayana1996, Dixon1998, Schobeiri2004, Hill1991, Logan2003, Baskharone2006}. Due to its importance, this equation will be briefly derived here.

From the conservation of angular momentum, 

\begin{equation}
    \vec{\acs{torque}} = \dot{m} (\vec{r}_{\text{out}}\cross\vec{V}_{\text{out}} - \vec{r}_{\text{in}}\cross\vec{V}_{\text{in}}) 
\end{equation}

Multiplying both sides by the rotational speed ($\omega$)

\begin{equation}
    \acs{power} = \vec{\omega}\cdot\vec{\acs{torque}} 
                = \dot{m} \vec{\omega} \cdot (\vec{r}_{\text{out}}\cross\vec{V}_{\text{out}} - \vec{r_{\text{in}}}\cross\vec{V}_{\text{in}}) 
\end{equation}

Assuming that the system is adiabatic and using the first law of thermodynamics, the specific enthalpy change is given by

\begin{equation}
    \Delta h = \frac{\acs{power}}{\dot{m}} 
             = \vec{\omega} \cdot (\vec{r}_{\text{out}}\cross\vec{V}_{\text{out}} - \vec{r}_{\text{in}}\cross\vec{V}_{\text{in}}) 
\end{equation}

\begin{equation}
    \Delta h = \omega (r_{\text{out}} {V_u}_{\text{out}} - r_{\text{in}} {V_u}_{\text{in}}) 
\end{equation}

\begin{equation}
    \label{eqn:euler}
    \Delta h = U_{\text{out}} {V_u}_{\text{out}} - U_{\text{in}} {V_u}_{\text{in}} 
\end{equation}

\subsection{Euler's equation for a centrifugal compressor}

It is useful to write the absolute tangential velocity at the compressor exit (${V_u}_3$)
as a function of the radial relative velocity (${W_r}_3$),
since the latter is more closely related to the mass flow 
(i.e. $\dot{m} = \rho {W_r}_3 A_3$, where $A_3$ is the ring shaped exit area of the compressor)

From the velocity triangle in figure X, assuming that the flow exits tangential to the blade surface,
\begin{equation}
    \label{eqn:V_u_3}
    {V_u}_3 = U_3 - {W_r}_3 \tan\beta_3
\end{equation}

Due to aerodynamic blockage at the tip and flow distortions caused by viscous efects, flow diffusion, clearences and blade aspect ratio,
${V_u}_3$ is actually smaller than what is predicted by \cref{eqn:V_u_3}.
This is accounted for by the \acf{impeller_distortion_factor}\index{impeller distortion factor}.
Furthermore, the flow does not exit perfectly tangential to the blade surface, and this is accounted by the \acf{slip_factor} \cite{Wiesner1967,Aungier1995}.
The corrected equation for the absolute tangential velocity is then

\begin{equation}
    \label{eqn:V_u_3_corrected}
    {V_u}_3 = \acs{slip_factor}(U_3 - \acs{impeller_distortion_factor}{W_r}_3 \tan\beta_3)
\end{equation}

Substituting \cref{eqn:V_u_3_corrected} in \cref{eqn:euler} and dividing both sides by ${U_3^2}$, we have

\begin{equation}
    \acs{load_coef}_c = \acs{slip_factor}(1-\acs{impeller_distortion_factor}\acs{flow_coef}_3\tan\beta_3) - \frac{U_2 {V_u}_2}{U_3^2} 
\end{equation}

where $\phi_3 \triangleq \frac{\dot{m}}{A_3 \rho_3 U_3} = \frac{W_{3r}}{U_3}$ is the flow parameter\index{flow parameter} at the compressor exit
and $\psi_c \triangleq \frac{\Delta h}{U_3^2}$ is the compressor power coefficient.

The second term accounts for pre-whirl\index{pre-whirl} in the flow entering the compressor, e.g due to \ac{IGV}.
In the case of no pre-whirl, ${V_u}_2 = 0$, thus 

\begin{equation}
    \label{eqn:euler}
    \boxed{\acs{load_coef}_c = \acs{slip_factor}(1-\acs{impeller_distortion_factor}\acs{flow_coef}_3\tan\beta_3)}
\end{equation}

\subsection{Euler's equation for an axial turbine}

\begin{figure}
    \sffamily\small 
    \caption{Velocities in an axial turbine stage}
    \label{fig:turbine_euler}
    \hfill\includesvg{fig/turbine_euler}\hfill
    \source{author's figure}
\end{figure}

Applying \cref{eqn:euler} to an axial turbine stage (\cref{fig:turbine_euler}), we have
\begin{equation}
    \Delta_h = U_5V_5-U_4V_5
\end{equation}
The tangential fluid velocities can be expressed in terms of their axial components, i.e.\
\begin{align}
    V_{4u} &= V_{4x} \tan\alpha_4 \\
    V_{5u} &= U_5 + V_{5x} \tan\beta_5
\end{align}
Then,
\begin{equation}
    \Delta h = U_5^2 + U_5 V_{5x}\tan\beta_5 - U_4 V_{4x}\tan\alpha_4
\end{equation}
And finally, making the equation non-dimensional by diving both sides by the blade speed at the exit $U_5$, we get
\begin{equation}
    \boxed{\psi_t = 1 + \phi_5\tan\beta_5-\left(\frac{r_4}{r_5}\right)^2\phi_4\tan\alpha_4}
\end{equation}
where $\phi_4 = \frac{V_{4x}}{U_5}$ and $\phi_5 = \frac{V_{5x}}{U_5}$ are the flow parameters and
$\psi_t=\frac{\Delta h}{U_5^2}$ is the power coefficient for the turbine.

\section{Loss models for centrifugal compressors}
\label{sec:compressor_losses}
Euler's equation is only valid for isentropic flow. 
For a real flow, corrections must be introduced to account for the generation of entropy in many parts of the compression process. 
These so called ``losses'' are modelled in literature as enthalpy corrections 
for the Euler equation considering both the \emph{actual} process and the \emph{isentropic} process 
that would generate the same increase in pressure as the real process.
Each of these losses can be categorized as either \emph{internal} or \emph{parasitic}.
Internal losses are those that happen inside the fluid main flow, and result in a decrease in the resulting pressure ratio.
Parasitic losses are the ones who happen outside the fluid main flow, 
and result in a increase in the temperature ratio through the compressor \cite{Galvas1973}.
Using these definitions, the actual and isentropic enthalpy changes through the compressor can be calculate as follows

\begin{align}
    \Delta h_{\text{actual}} &= \Delta h_{\text{euler}} + \sum_{\substack{
                                                            \text{parasitic} \\ 
                                                            \text{losses}
                                                        }}
                                                          \Delta h \\
    \Delta h_{\text{isen}}  &= \Delta h_{\text{euler}} - \sum_{\substack{
                                                            \text{internal} \\ 
                                                            \text{losses}
                                                        }}
                                                          \Delta h
\end{align}

The temperature and pressure ratios are given by
\begin{align}
    \frac{T_{03}}{T_{02}} &= 1 + \frac{\Delta h_{\text{actual}}}{c_p T_{02}} \\
    \frac{P_{03}}{P_{02}} &= \left( 1 + \frac{\Delta h_{\text{isen}}}{c_p T_{02}}\right)^{\frac{\gamma}{\gamma-1}}
\end{align}

The isentropic and polytropic efficiencies can then be readly calculated as
\begin{align}
    \eta_t &= \frac{\Delta h_{\text{isen}}}{\Delta h_{\text{actual}}}\\
    \eta_p &= \frac{\gamma-1}{\gamma} \left(\frac{\log\frac{P_{02}}{P_{01}}}
                                                 {\log\frac{T_{02}}{T_{01}}}
                                     \right)
\end{align}

\textcite{Aungier1995} criticises this calculation because the losses should be regarded as changes in entropy, and not in enthalphy. Despite this shortcomming, the enthalphy approach is widely used.

The losses generally accounted for are listed in \cref{tbl:compressor_loss_mechanisms}. According to \textcite{Gravdahl2004,Gravdahl1999,Ferguson1963,Watson1982}, the most important losses for stability considerations are the incidence losses and the skin friction losses. The former is destabilizing because it introduces the positive slope of the left hand side of the compressor map. The fluid friction losses are stabilizing because they increase with mass flow, therefore contributing to the decrease in the derivative of the iso-speed lines.
\begin{table}
\caption{Loss mechanisms for a centrifugal compressor}
\label{tbl:compressor_loss_mechanisms}
\hrule
\begin{multicols}{2}
\begin{compactitem}
    \item[] \textbf{Internal}
    \item Incidence loss (in the impeller and diffuser)
    \item Blade loading loss
    \item Skin friction loss
    \item Clearance loss
    \item Mixing loss
    \item Vaneless diffuser loss
    \columnbreak
    \item[] \textbf{Parasitic}
    \item Disc friction loss
    \item Recirculation loss
    \item Leakage loss
\end{compactitem}
\end{multicols}
\hrule
\source{\cite{Oh1997}}
\end{table}

A detailed discussion of each loss mechanism is beyond the scope of this work; 
the interested reader should refer to \textcite{Cumpsty2004}. 

Semi-empirical models are available to predict each individual loss based on compressor geometry and flow conditions. 
In particular, \textcite{Oh1997} did a review of various loss models for each loss mechanism and selected an optimal set.

\section{Loss models for axial turbines}

Most of the empirical loss correlations for axial turbines in use today stem from the seminal work by \textcite{Ainley1951}. This work identified the main loss mechanisms for an axial turbine stage as 
\begin{description}
    \item[profile loss] due to skin friction and flow separation under high incidence angles of the blades
    \item[annulus loss] due to boundary layer growth on the inner and outer walls of the turbine annulus
    \item[secondary flow loss] due to the tri-dimensional nature of the flow
    \item[tip clearence loss] due to the vortex formed at the tip of the turbine blades
\end{description}

Several improvements were made to Ainley and Mathieson's model by \textcite{Mukhtarov196, Dunham1970, Kacker1982, Moustapha1990, Benner1995} among others\cite{Persson2015}. All of these studies were based on empirical data available from previous turbine rig testing, and thus are valid for turbines with similar geometries.
These models provide the loss values as either a pressure loss-coefficient
\begin{equation}
    Y = \frac{P_{04}-P_{05}}{P_{05}-P_5}
\end{equation}
or a kinectic energy loss coefficient
\begin{equation}
    \phi^2 = \left(\frac{\text{actual gas exit velocity}}
                        {\text{ideal gas exit velocity}}\right)^2
\end{equation}

\section{Deviation models for axial turbines}

The flow exiting a turbine blade row does not follow the blade camber line exactly. 
This is analogous to the slip observed in centrifugal compressors, but needs more elaborate correlations, 
thus deserving a section of its own. 
The simplest correlation available is known as Carter's rule \cite{Mattinlgly1995}, and states that
\begin{equation}
    \label{eqn:cartersrule}
    \delta = \frac{\beta_4-\beta_5}{4\sqrt{\sigma}}
\end{equation}

where $\delta$ is the angular deviation between the exit flow and the exit camber angle,
$\beta_4$ and $\beta_5$ are the inlet and exit airfoil camber angles,
and $\sigma=\frac{\text{chord}}{\text{blade separation}}$ is the solidity.

Another model based on inlet and exit Mach numbers is provided by \textcite{Ainley1951}, 
and a more elaborate model involving also blade thickness was proposed by \textcite{Islam1999}, where
\begin{equation}
    \delta = \frac{\frac{\rho_5 V_{x4}}{\rho_5 V_{x5}} \sigma^{-1.1} (\alpha'_4+\beta_5)^{2.25}}
                        {\xi^{1.45} \left(t_{\max}/c\right) \left(22+0.22\beta_4^{1.64}\right)}
\end{equation}
where $\xi$ is the blade geometric incidence (stagger angle), 
$t_{\max}/c$ is the blade maximum thickness-to-chord ratio 
and all angles are given in degrees.


\section{Engine simulation (dynamic and static)}

The engine model used is of the lumped volume type. This means that each component 
 (compressor, combustor,turbine and nozzle)
 is modelled as a single point in space and conservation laws are applied to them.
 This type of model is also known as zero dimensional (0-D) or as parametric cycle analysis
 \index{parametric cycle analysis},
 and is widely used in both industry and academy as a tool for preliminary design and 
 performance analysis. 

A schematic of the engine components considered for the simulation of the present gas turbine engine is shown in \todo{add figure}.

\subsection{Turbojet thermodynamic cycle}


\begin{figure}
    \sffamily\small 
    \caption{Schematic diagram of the modelled engine}
    \hfill\includesvg{fig/engine_schematic}\hfill
    \source{author's figure}
\end{figure}

The thermodynamic cycle of gas turbines is the subject of many books \cite{Mattingly1996, Cumpsty2015}, 
and will only be discussed here in brief.

\subsection{What is a component map/characteristic}

A component map or characteristic is, in a general sense, the functional relationship between a component's
mass flow rate, rotation speed, pressure ratio and efficiency. 
These five unknowns are linked by two constraints, namely Euler's equation and a loss model. 
This leaves two free variables.
Usually compressor maps are plotted using the mass flow rate as the abscissa and  the pressure ratio as the ordinate, with contours for constant rotation speed and efficiency. 
For turbines, the rotational speed and efficiency is not uniquely defined for a combination of mass flow rate and pressure ratio. This led to the use of the artificial parameter of rotational speed times mass flow rate for the abscissa.

The name characteristic is most often used in reference to the iso-speed lines of a turbine or compressor when plotted in the plane of mass flow vs.\ pressure ratio.

Another way of plotting compressor maps is using the flow parameter as the abscissa and  the pressure rise or efficiency as ordinate. 
For incompressible flow this leads to the collapse of the constant rotational speed lines, thus being very convenient.


\subsection{Available programs}

There are many readily available computer programs that implement this kind of simulation. 
From these, the most famous is probably the commercial program GasTurb \cite{GasTurb}, 
\index{software} \index{GasTurb}
followed by the \gls{GSP} \cite{Visser2000}.
A comparison of both is presented in \textcite{GasTurbvsGSP}.
In particular, GasTurb has already been used to simulate model gas turbines 
\cite{gao2011modelling}.
An open-source alternative is the \gls{T-MATS}, from NASA \cite{T-MATS}.
\gls{T-MATS} provides a library of turbo machinery blocks for use in Simulink. 
Each of these blocks is actually a wrapper for a function written in C 
that simulates the component's behaviour.

The downside of all these programs is that they rely on component map data provided by the user.
The generation of the maps is a substantial effort that must be carried independently, 
either by scaling maps from similar designs, obtaining the maps experimentally 
or using semi-empirical methods as described in this work. 
In most cases, if a new map has to be developed from scratch, a mixture of semi-empirical and
experimental data will be used.
This is because it is hard to operate the components in all points of the map while in a test bed, 
and experimental measures are inherently noisy. 
It is therefore best to use this data to calibrate semi-empirical maps.

\section{Nondimensional parameters}

\section{Surge}
\label{sec:review:surge}

Surge is a phenomenon related to the stability of dynamical flow systems which involve a compressor. 
It is characterized by periodic backflow accompanied by vibration and a characteristic sound, 
which can be a series of bangs for engines with large internal volume or a low frequency hum for smaller engines.
In extreme cases, it can cause damage to the engine and will always cause a significant reduction in mass flow and thrust. 
Normally the system can not recover from surge on its own, and either a reduction in fuel flow or the opening of bleed valves is required to return the engine to its steady state.
Although the surge is intimately related to the compressor, 
it is also dependent on the engine components downstream from it \cite{Sparks1983}.

A simple model for surge with all the fundamental elements is due to \textcite{Finsk}. This model includes a compressor, a plenum of known volume, some plumbing of known length and an adjustable throttle valve. 
In an actual engine, the plenum and plumbing are analogous to the combustion chamber, while the throttle can be considered as a simplified turbine-nozzle assembly. 
The stability criterion derived from this model by linear control theory is

\begin{align}
    C' &< \frac{1}{B^2 T'} \label{eqn:surge} \quad\text{and}\\
    C' &< T'
\end{align}
where $C' = \frac{\partial \pr{03}{02}}{\partial \dot{m}_c}$ is the derivative of the compressor characteristic, $T' = \frac{\partial \pr{05}{04}}{\partial \dot{m}_t}$ is the derivative of the turbine characteristic and 
\begin{equation}
    B=
\end{equation}

\textcite{Gravdahl} showed good agreement of this model with practice.

This model provides a good physical explanation of the mechanism of surge. 
First, one should notice that, by the first stability criterion, surge can only occur when the compressor characteristic has a positive derivative, 
i.e.\ when an increase in mass flow is accompanied by an increase in the pressure ratio supplied by the compressor. 
If this increase in pressure ratio does not lead to an increased mass flow through the throttle, the increase in compressor mass flow ration will increase the plenum pressure above that which the compressor can supply, and this will result in backflow. In the negative part of the compressor characteristic, it will be able again to supply the demanded pressure, now made smaller due to backflow. Since this condition will lead to an ever decrease in the plenum pressure, the flow will eventually invert its direction again, and this results in the periodic behaviour of surge.

Analysing the extreme values of the parameter $B$ is helpful in developing an intuitive grasp for the phenomenon.
For large values of $B$, that is, large plenum volumes, the surge limit approaches the point where the compressor characteristic has zero derivative. Since the plenum is very big, any change in the mass flow of the throttle will make a very small difference in the plenum pressure, and so any simultaneous increase in compressor mass flow and pressure ratio will lead to instability. This is the conservative limit for the surge line.
Inversely, for small values of $B$, i.e.\ approaching unit, any increase in pressure will lead to an increase in mass flow through the throttle, which will instantaneously reduce the pressure, thus having an stabilizing effect. 
In this case, the two criteria for surge collapse into one.


Another result from this model is that while surge is often related to compressor stall, this is not necessarily the case.
The surge line for a compressor system can be moved be changing components downstream from it, and operation of compressors beyond the surge line using control systems based on bleed valves is widely reported in the literature \cite{bla}, and that would not be possible if surge was simply an aerodynamical effect.

The value of the nondimensional parameter $B$ is always greater than 1. 

\section{Numerical aspects}
Closing the loop one variable at a time as suggest by \textcite{walsh2004gas} is not the ideal way to solve this systems. This method is not only hard to set up and dependent on quite a lot of intuition acquired through experience, but it is also very prone to diverge (sometimes wildly so) if the updated values for the variables are not chosen just right. A more robust approach is to use proven solvers for non-linear systems.

\subsection{Nonlinear solvers}
\subsubsection{Gradient based solvers}
\subsubsection{Homotopic solvers}
\subsection{\ac{ODE} solvers}
\subsubsection{Explicit solvers}
\subsubsection{Implicit solvers}
\subsection{Graph theory}
\subsubsection{Viewing a grid/mesh as a graph}
\subsubsection{Breadth first search}


\chapter{Methods}
\label{sec:methods}
\epigraph{It's not denial. \\ I'm just very selective about the reality I accept.}{-Calvin}
Assumptions: ideal gas
\section{Compressor model}

The incidence loss model used assumes that all kinectic energy from the velocity components
normal to the inducer and diffuser blades is lost \cite{Stanitz1953}.
All other loss models used were suggested by the comparative study of \textcite{Oh1997}, 
and will not be discussed here.

\subfile{text/body/compressor_map}

From the Buckinham $\Pi$ Theorem \cite{Buckingham1914}, 
it can be shown that the compressor behaviour is a function of only three parameters, 
namely the \acl{MFP}, the \acl{M0rotor}, the \acl{gam} and the \acl{Re}. 
For a centrifugal compressor, this quantities are defined respectively as follows

\begin{align}
    \acs{MFP} &\triangleq \MFP{1} = \MFPalt{1} \\
    \acs{M0rotor} &\triangleq \frac{U_3}{\sqrt{\gamma R T_{01}}} \\
    \gamma &\triangleq \frac{c_p}{c_v} \\
    \acs{Re} &
\end{align}

In particular, the \acl{MFP} is a function of the Mach number and the specific heat ratio,
 i.e.\
\begin{equation}
    \label{eqn:mfp2mach}
    \acs{MFP}(M,\gamma) = M \left( 1 + \frac{\gamma-1}{2}M^2\right)^{-\frac{\gamma+1}{2(\gamma-1)}}
\end{equation}
While there is no closed form inverse for this function, it is invertible and the inverse can be easily computed numerically, e.g.\ by a newton scheme \cite{Der1974}.

According to \textcite{walsh2004gas},
 the influence of the \acl{Re} in the system behaviour is only of second order.

To write Euler's equation \cref{eqn:euler} in terms of these parameters, suffices to rewrite $\phi_3$ and $\psi$ as a function of them. Thus
\begin{multline}
    \label{eqn:phi2_dimensionless}
    \phi_3 = \frac{W_{2r}}{U_3} 
           = \frac{\frac{\dot{m}}{\rho_3 A2}}{\acs{M0rotor} a_{02}}
           = \frac{\MFPalt{2}}{\acs{M0rotor}} \frac{a_{03}}{a_{02}} \stagdensratio{3} \\ 
           = \frac{\acs{MFP}_3}{\acs{M0rotor}} \sqrt{\frac{T_{03}}{T_{01}}} \stagdensratio{3}
\end{multline}
The value of $M_3$ can be obtained from $\acs{MFP}_3$ using \cref{eqn:mfp2mach}, while the latter is given by
\begin{equation}
    \label{eqn:mfp2mfp2}
    \acs{MFP}_3 \triangleq \MFP{2} = \acs{MFP} \frac{A_2}{A_3} \frac{P_{02}}{P_{03}} \sqrt{\frac{T_{03}}{T_{02}}}
\end{equation}
Notice that \cref{eqn:mfp2mfp2} makes no assumption of the process 1--2 being isentropic. $\acs{MFP}_3$ can be used along with \acs{MFP} to check for choking.

The load coefficient is given by
\begin{equation}
    \acs{load_coef}\triangleq \frac{\Delta h}{U_3^2}
                      = \frac{c_p(T_{02}-T_{02})}{U_3^2}
                      = \frac{\frac{c_p(T_{03}-T_{01})}{\gamma R T_{02}}}
                                    {\acs{M0rotor}^2}
                      = \frac{\frac{T_{03}}{T_{02}}-1}
                                  {\acs{M0rotor}^2}
                        \left(\frac{1}{\gamma-1}\right)
\end{equation}

At this point the loss models discussed in \cref{sec:compressor_losses} are introduced, 
and the values of $\acs{load_coef}_{\text{actual}}$ and $\acs{load_coef}_{\text{isen}}$ are obtained.
The temperature and pressure ratios can then be readily calculated from
\begin{align}
    \label{eqn:psi2Tratio}
    \frac{T_{03}}{T_{02}} &= (\gamma - 1)\psi_{\text{actual}} \acs{M0rotor}^2 + 1 \\
    \label{eqn:psi2Pratio}
    \frac{P_{03}}{P_{02}} &= \left[(\gamma - 1)\psi_{\text{isen}} \acs{M0rotor}^2 + 1\right]^\frac{\gamma}{\gamma-1}
\end{align}

In this work, only the losses due to incidence and skin friction were considered, because they are the ones most important for compressor stability (see \cref{sec:compressor_loss}). The model chosen for skin friction was suggest by \textcite{Oh1997} and the incidence losses where modelled as the complete conversion of all kinetic energy normal to be blade camber line to heat \cite{Galvas1973}. This is a more phisical model than the one suggested by Oh et al., and captures better the loss variation with incidence, since Oh et al's model does not include flow angles. 


\section{Turbine model}
\subfile{text/body/turbine_map}

The turbine model is sumarized in \cref{map:turbine}. 
Its development is mostly analogous to that of the compressor, with the exception of the loss and deviation models. 
Since the turbine does not exhibit flow instability phenomena such as compressor surge, 
which depends heavly on the incidence losses, 
and this work is focussed on the dynamic moddeling of the engine, 
the turbine was assumed isentropic. 
Nevertheless, a loss model based on pressure loss coefficients $Y_p$, $Y_s$, $Y_k$, etc.\ 
could easily be included in this model by modfying \cref{eqn:turb_res_pr} in the following way
\begin{equation}
    \frac{P_{05}}{P_{04}} -\frac{[1 + (\gamma-1)\psi_{\text{euler}} M_b^2]^{\frac{\gamma}{\gamma-1}}}{1+Y\left\{1-\left[1+\tfrac{\gamma-1}{2}M_5^2\right]^{-\frac{\gamma-1}{\gamma}}\right\}} = 0 
\end{equation}

where
\begin{equation}
    Y = Y_p + Y_s + Y_k
\end{equation}

This correction is derived as follows. The pressure loss coefficient is given by
\begin{equation}
    Y = \frac{P_{0i} - P_{0e}}{P_{0e} - P_e}
\end{equation}
Now we divide the numerator and the denominator by $P_{0e}$ and isolate the pressure ratio term 
\begin{equation}
    \frac{P_{0i}}{P_{0e}} = \frac{1}{1+Y\left\{1-\left[1+\tfrac{\gamma-1}{2}M_e^2\right]^{-\frac{\gamma-1}{\gamma}}\right\}}
\end{equation}
This pressure ratio loss term must be multiplied to the ideal (isentropic) pressure ratio given by the second term of \cref{eqn:turb_res_pr}. If more than one blade row is considered (e.g.\ one for the stator and another one for the rotor) the pressure ratio losses for each must be joined by multiplication.

\subfile{text/body/nozzle}


\section{Engine static model}


The engine static model, also known as steady state, 
describes the conditions in which the engine can operate indefinitelly 
without changes in throttle input or outside conditions (pressure and temperature).


Equations:
\begin{enumerate}
    \item Constant spool speed
    \item Conservation of energy transmited through spool
    \item Conservation of mass in the burner
    \item Energy adition due to fuel
    \item nozzle
    \item turbine (x3)
    \item compressor (x3)
\end{enumerate}

\subfile{text/body/engine_map}

\subsection{Performance parameters}
\begin{itemize}
    \item thrust
    \item thermal efficiency
\end{itemize}

\section{Engine dynamical model}
\begin{align}
    \dot{P}_{04} &= \frac{a_{01}^2}{\volume}(\dot{m}_c - \dot{m}_t) \\
    \ddot{m}_c &= \frac{A_1}{L_c}(P_{03}-P_{04}) \\
    \dot{\omega} &= \frac{1}{J} (\tau_t-\tau_c)
\end{align}

\section{Surge model}

The model due to \textcite{Finsk}, described in \cref{sec:review:surge} was adopted here. 
Since component matching is an important part of the surge mechanism, the engine matching model equations were mostly replicated here, with the exception of the nozzle equation since the exit pressure of the nozzle can be allowed to be arbitrary to consider all possible surge conditions. Surge will normally occur when the mass flow is severely reduced, which happens when the overall engine pressure ratio is big. This equation was replaced by Fink's stability criterion \cref{eqn:surge}. The resulting model is in \cref{model:surge}.

\section{Numerical aspects}
% Literary review (TODO: move this)
As will be shown in \cref{sec:methods}, 
the moddeling of a gas turbine engine and of its components can 
be conveniently described by nonlinear algebraic systems. 
There are two main approaches to solve these systems numerically: 
classical gradient based root finding and optimization 
and path following algorithms (\ac{DAE} solvers).

% Methods
\subsection{Generation of component maps}
Nonlinear solver (hybr or lm) coupled with a breadth-first search. Initial point in the $x$ and $y$-axis is chosen randomly within the choke boundaries, pressure and temperature ratios are unit, mach numbers are between 0 an 1. If convergence is achieved a breadth-first search considering that each point in the grid is a graph node connected to the points up, down, left, right and in the diagonals. The converged result for each point is used as an initial guess for neighboring points. This allows each point in the grid up to 8 different initial guesses quite close to the actual solution (if the grid is fine enougth), and has proven very robust. 
\subsubsection{choke line}
Compressor: find point where inlet and exit are simultaniously choked. For pressure ratios above this critical value, only the inlet is choked and the choke line is vertical. For pressure ratios bellow this value the exit is choked. The choke line was verified to be a straight line, so that just the point at another pressure ratio, e.g.\ the minimum pressure ratio of interest (normally unit) is needed to fully define it.

Turbine: it is linear in a plot of \acs{MFP} vs.\ pressure ratio, but in this case the blade mach number is not uniquelly defined. In a plot of $\acs{MFP}M_b$ vs.\ expansion ratio, the choke line is highly non linear and a KS aggregation function with variable $rho$ was added to the turbine model to model the choke.  

\subsection{Generation of the working line}
Optimize the engine model for maximum efficiency. 
Then use a \ac{DAE} solver to continue the line forwards and backwards.

\subsection{Dynamical simulation}
Inputs: step in T04 up and down; senoids?

\section{The VT-80 engine}
The engine chosen to which apply these models was Jet-Munts VT-80.

\chapter{Results and discussion}
\epigraph{Que tanto de coisa para estragar.\\{\footnotesize That's a lot of parts to need fixing.}}{- My Grandfather about my dad's new car}

\section{Engine matching}
\begin{sidewaysfigure}[p]
    \AddThispageHook{\thispagestyle{empty}}
    \caption{Component maps and work line of the VT-80 engine}
\includegraphics{fig/wline.pdf}
    \source{author's figure}
    \legend{Compressor and turbine maps for the VT-80 engine, with the work line superimposed (thick line).
    The medium thickness contours are of constant blade mach number and the thin countours are of constant polytropic efficiency.}
\end{sidewaysfigure}

\begin{figure}
    \centering
    \caption{Flow state at each engine station}
    \includegraphics{fig/stations1000K}
    \source{author's figure}
    \caption*{Temperature, pressure and mach number for each engine station at a turbine inlet temperature of 1000K and flight mach number zero. The temperature and pressure scales were chosen to keep their ratios to the reference values $\mathsf P_{01}$ and $\mathsf T_{01}$ comparable. When the station number is prefixed with a ``0'', the properties are from stagnation, otherwise they are static. Interpolation was done with the PCHIP algorithm \cite{Fritsch1980, Moler2004}, to keep lines smooth while not introducing artificial maxima.}
\end{figure}

Compressor and turbine show a nice match, even though many of the losses are not included in the model.
This is probably due to the overly optmistic efficiencies in both the compressor and the turbine compensating for each other.

turbine has a poor design because it tends to choke on exit instead of on inlet.

nozzle seems to be too big for the engine. In the simulation it was reduced from the measured diameter of 45mm to 35mm. This may be due to the overestimation of the components' efficiencies.

\begin{figure}
    \caption{Matching sensitivity to nozzle diameter}
\end{figure}

\section{Dynamical simulation}
\subfile{text/body/conclusion}

\end{document}
