% ---
% RESUMOS
% ---

\setlength{\absparsep}{18pt} % ajusta o espaçamento dos parágrafos do resumo
\begin{resumo}
    The simulation of gas turbine engines is of vital importance to their design process and to the design of their control systems.
    In order to capture behaviour not shown by a simple thermodynamic cycle model, such as transients, unchoked turbine inlet, and instabilities, 
    as well as to increase the model fidelity, it is necessary to develop a component based engine model. 
    This work develops such engine model  for the simple case of a single spool turbojet, 
    with a single stage centrifugal compressor and single stage axial turbine. This configuration is typical of very small gas turbines.
    No experimental data is need as input to the model. It is based on first principles and semi-empirical corrections available in the literature.
    This work contains nonlinear models for a centrifugal compressor, axial turbine, nozzle, and the engine operating statically and dynamically.
    Surge is considered. The models are applied for the Jet Munts VT-80 engine and the results are compared with experimental data.

\end{resumo}

\begin{resumo}[Resumo]
 \begin{otherlanguage*}{brazil}
     A simulação de turbinas a gás é essencial para seu projeto e para o projeto dos seus sistemas de controle .
     Para simular características deste tipo de motor que não são capturadas por um simples modelo de ciclo termodinâmico, 
     e também para aumentar a fidelidade da simulação, é necessário desenvolver um modelo de componentes.
     Este trabalho desenvolve um modelo desse tipo para o caso de um motor turbojato de um único eixo, 
     com um compressor centrífugo de um estágio e uma turbina axial também de um único estágio.
     Essa configuração é típica para mini-turbinas a gás.
     Nenhum dado experimental é necessário como entrada para o modelo, pois ele é baseado em leis de conservação e correções semi-empíricas encontradas na literatura.
     Este trabalho contém modelos não lineares para compressores centrifugos, turbinas axiais, bocais, e motores completos operando em regime permanente ou transiente.
     A purga do compressor é levada em conta. Esses modelos são aplicados para o caso do motor Jet Munts VT-80 e os resultados comparados com dados experimentais.
     
   \noindent 
 \end{otherlanguage*}
\end{resumo}

% ---
