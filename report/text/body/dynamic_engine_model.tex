\documentclass[tcc]{subfiles}

\begin{document}

\chapter{Dynamic engine model}
\label{ch:engine_model}
\epigraph{\em ``It's not denial.\\ I'm just selective about the reality I
accept.''}{\em(Bill Watterson)}

To develop a controller for any kind of system,
 the first thing needed is a model which reproduces the system's behavior in time.
 i.e.\ a dynamic or transient model.
 For the present work, we must model a gas turbine engine.

\section{Model description}
The engine model used is of the lumped volume type. This means that each component 
 (compressor, turbine, combustor and so on)
 is modelled as a single point in space and conservation laws are applied to them.
 This type of model is also known as zero dimensional (0-D) or as parametric cycle analysis.
 The components considered for the simulation of the present gas turbine engine are shown in \todo{add figure}.

\todo{Literature review of different forms of simulation}

Somewhat surprisingly,
 the simplest space state model of a single shaft turbine has only one state and one input.
 The input is fuel flow and the state is the shaft rotational speed.
 Every other parameter can be derived from this two by an off-design calculation
 which allows for excess power in the turbine.
 Since this off-design calculation is also needed to determine the derivative of the rotational speed,
 and it contains all the thermodynamic of the engine, it makes the system highly non-linear.

The turbomachinery in the engine are characterized using component maps.\index{Component map}
 These are essentially lookup tables which relate 
 rotation speed, corrected temperature, corrected mass flow and efficiency
 for the turbine and compressor.

\todo{Walkthrough of one N dot calculation}
\todo{Figure: transient calculation flowchart}

\section{Architecture for computer simulation}
\section{Experimental parameter estimation}

\end{document}
