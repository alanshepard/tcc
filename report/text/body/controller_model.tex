\documentclass[tcc]{subfiles}

\begin{document}

\chapter{Controller model}
\label{ch:control}
\epigraph{\em ``Keep it simple:\\
as simple as possible,\\
but no simpler.''}{\em (Albert Einstein)}

\section{Requirements}
The controller shall
\begin{itemize}
    \item start the engine.
          In case of a failed attempt at starting, it must purge the combustion chamber;
    \item prevent surge during operation;
    \item accelerate and decelerate the engine according to pilot input in a timely fashion;
          \todo{be more specific}
    \item bring the engine to idle and shut it down when the pilot commands it;
    \item detect self extinction and cut-off the fuel supply.
    \item limit the turbine entry temperature to \todo{1200K}
\end{itemize}

\section{Controller architecture}
\begin{figure}[tp]
    \centering
    \caption{State transition diagram for the \acs{FADEC}}
    \tikzstyle{decision} = [diamond, draw, 
    text width=4.5em, text badly centered, node distance=3cm, inner sep=0pt]
\tikzstyle{block} = [rectangle, draw, 
    text width=5em, text centered, rounded corners, minimum height=4em]
\tikzstyle{line} = [draw, -latex']
\tikzstyle{state} = [draw, ellipse, %node distance=3cm,
    minimum width =3em, text badly centered, minimum height=3em]
    
\begin{tikzpicture}[node distance = 2.5cm, auto]
    %% Place nodes
    % normal path
    \node [state,initial above] (shutdown) {Shutdown};
    \node [state, below of= shutdown] (dry_crank) {Dry crank};
    \node [state, below of=dry_crank] (lights_off) {Lights off};
    \node [state, right of=lights_off, node distance = 16em] (purge) {Purge};
    \node [state, below of=lights_off] (acceleration) {Acceleration};
    \node [state, below of=acceleration] (heat_soak) {Heat soak};
    \node [state, below of=heat_soak, text width=4.5em] (normal) {Normal operation};
    \node [state, left of=acceleration, node distance=12em] at ($(normal)!0.5!(shutdown)$) (stable_idle) {Stable idle};
    
    % emergency shutdown
    %\node at (current bounding box.south east) [anchor=south east, state, text width=4.5em] (emergency_shutdown) {Emergency shutdown};
    %\node [left of=emergency_shutdown, node distance=10em] (extinction) {};
    %\node [above of=emergency_shutdown, node distance =8em] (emergency_command){};
    
    %% Draw edges
    \path[->] (shutdown)     edge              node {on/off switch = on} (dry_crank)
              (dry_crank)    edge              node {N=x rpm}            (lights_off)
              (lights_off)   edge              node {no EGT increase}    (purge)
                             edge              node {EGT increase}       (acceleration)
              (purge)        edge [bend right]                           (shutdown)
              (acceleration) edge [bend right]                           (heat_soak)
              (heat_soak)    edge [bend right]                           (acceleration) 
                             edge              node{N = ground idle}     (normal)
              (normal)       edge [bend left]  node[text width=4.5em]{shutdown command}    (stable_idle)
              (stable_idle)  edge [bend left]                            (shutdown)
              %(extinction)    edge              node{extinction}          (emergency_shutdown)
              %(emergency_command) edge          node[text width=4.5em]{emergency shutdown command} (emergency_shutdown)
    ;

    %% Rectangles
    %\draw (dry_crank.north west) rectangle (lights_off.south east);
\end{tikzpicture}

    
    \source{\authorsfigure}
    \label{fig:controller_state_machine}
\end{figure}
The controller is implemented as a state machine \cite{amd_state_machine} 
 as shown in \autoref{fig:controller_state_machine}. 
The allowed states are
\begin{description}
    \item[Shutdown] engine is off. 
    \item[Dry cranck] starter motor is on and engine rotates,
          but no fuel is supplied.
    \item[Lights off] the ignition per say. 
    \item[Acceleration and heat soak] after a successful ignition,
          the engine goes through an acceleration schedule that allows heat soaking. 
    \item[Normal Operation] when the engine is warm enough, it can be operated normally.
          This is the flight ready condition.
    \item[Purge] after an unsuccessful start, the engine must eliminate the unburnt fuel. 
    \item[Ground idle] before shutting down, the engine must be in a stable low rpm idle condition.
\end{description}


\section{Settings and gains}

\end{document}
