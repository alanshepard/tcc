\documentclass[tcc]{subfiles}

\begin{document}

\chapter{Introduction}
\label{ch:intro}
\epigraph{\em ``If you can't control your peanut butter,\\ 
                you can't expect to control your life.''
         }{\em(Bill Watterson)}

A \gls{FADEC} is the collection of control system elements in 
 a modern aeronautical engine \cite{rolls-royce}.
The purpose of a \gls{FADEC} is to minimize the workload of the pilot,
 while still giving him or her total control over the engine.
In this way, the pilot is responsible for selecting a power setting 
 and the \gls{FADEC} is then in charge of getting to that condition
 as fast and safely as possible. It is therefore responsibility of the
 control system to guarantee the operation of the engine within safe limits:
 it must prevent surges when accelerating 
 and combustion extinction when decelerating.

Virtually all modern gas turbines are controlled by a digital system.
The technology used in these systems, however, is proprietary most of times.
This thesis attempts to develop a \gls{FADEC} 
for a small (model sized) aeronautical engine for both technological demonstration 
and to increase the scarce literature on the subject.

% What you should know before reading this text
Since this report deals with the design of an engine control system,
 it is essential the reader to be familiar with both control theory and transient gas turbine dynamics.
For an introduction to control theory, the reader should refer to 
 \textcite{dorf2011modern,franklinfeedback}.
There are many texts that cover jet engine modelling, 
 albeit not many deal with starting and transient behaviour. 
For a complete treatment, the reader should refer to \textcite{walsh2004gas}. 
\textcite{rolls-royce} provides a very comprehensive 
 but not very mathematical description of jet engines. 
 It is a good book to develop an intutive feel for jet engine operation. 
Finally, \textcite{cumpsty2015jet} is a valuable introductory text on the subject.

%Outline
The rest of this thesis is organized as follows: 
\begin{description}
    \item[\Cref{ch:methods}] describes the engine and instrumentation used.
    \item[\Cref{ch:engine_model}] explains the transient model of the engine detail.
        Both the physics principles and its implementation are discussed. 
        This chapter introduces the concept of a component map, 
         essential for gas turbine simulation.
    \item[\Cref{ch:control}] covers both the design and implementation of the control system.
    \item[\Cref{ch:tests}] describes the experimental results of both the engine by itself 
         and integrated with the developed control system. 
        It contains the experimentally obtained maps for this engine's components.
    \item[\Cref{ch:conclusion}] presents the conclusion and final remarks of this work.
\end{description}

\end{document}
