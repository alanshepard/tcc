\documentclass[tcc]{subfiles}

\begin{document}
    % \pretextual

    % ---
    % Capa
    % ---
    \imprimircapa
    % ---

    % ---
    % Folha de rosto
    % (o * indica que haverá a ficha bibliográfica)
    % ---
    \imprimirfolhaderosto*
    % ---

    % ---
    % Inserir a ficha bibliografica
    % ---

    % Isto é um exemplo de Ficha Catalográfica, ou ``Dados internacionais de
    % catalogação-na-publicação''. Você pode utilizar este modelo como referência. 
    % Porém, provavelmente a biblioteca da sua universidade lhe fornecerá um PDF
    % com a ficha catalográfica definitiva após a defesa do trabalho. Quando estiver
    % com o documento, salve-o como PDF no diretório do seu projeto e substitua todo
    % o conteúdo de implementação deste arquivo pelo comando abaixo:
    %
    % \begin{fichacatalografica}
    %     \includepdf{fig_ficha_catalografica.pdf}
    % \end{fichacatalografica}

    \begin{fichacatalografica}
        \sffamily
        \vspace*{\fill}					% Posição vertical
        \begin{center}					% Minipage Centralizado
        \fbox{\begin{minipage}[c][8cm]{13.5cm}		% Largura
        \small
        \imprimirautor
        %Sobrenome, Nome do autor
        
	\hspace{0.5cm} \imprimirtitulo  / \imprimirautor. --
	\imprimirlocal, \imprimirdata-
	
	\hspace{0.5cm} \pageref{LastPage} p. : il. (algumas color.) ; 30 cm.\\
	
	\hspace{0.5cm} \imprimirorientadorRotulo~\imprimirorientador\\
	
	\hspace{0.5cm}
	\parbox[t]{\textwidth}{\imprimirtipotrabalho~--~\imprimirinstituicao,
	\imprimirdata.}\\
	
	\hspace{0.5cm}
		1. Palavra-chave1.
		2. Palavra-chave2.
		2. Palavra-chave3.
		I. Orientador.
		II. Universidade xxx.
		III. Faculdade de xxx.
		IV. Título 			
	\end{minipage}}
	\end{center}
\end{fichacatalografica}
% ---

% ---
% Inserir folha de aprovação
% ---

% Isto é um exemplo de Folha de aprovação, elemento obrigatório da NBR
% 14724/2011 (seção 4.2.1.3). Você pode utilizar este modelo até a aprovação
% do trabalho. Após isso, substitua todo o conteúdo deste arquivo por uma
% imagem da página assinada pela banca com o comando abaixo:
%
% \includepdf{folhadeaprovacao_final.pdf}
%
\begin{folhadeaprovacao}

  \begin{center}
    {\ABNTEXchapterfont\large\imprimirautor}

    \vspace*{\fill}\vspace*{\fill}
    \begin{center}
      \ABNTEXchapterfont\bfseries\Large\imprimirtitulo
    \end{center}
    \vspace*{\fill}
    
    \hspace{.45\textwidth}
    \begin{minipage}{.5\textwidth}
        \imprimirpreambulo
    \end{minipage}%
    \vspace*{\fill}
   \end{center}
        
   This dissertation was aprooved on \today, by:

   \assinatura{\textbf{\imprimirorientador} \\ Supervisor} 
   \assinatura{\textbf{Professor} \\ Reader 1}
   \assinatura{\textbf{Professor} \\ Reader 2}
   %\assinatura{\textbf{Professor} \\ Reader 3}
   %\assinatura{\textbf{Professor} \\ Reader 4}
      
   \begin{center}
    \vspace*{0.5cm}
    {\large\imprimirlocal}
    \par
    {\large\imprimirdata}
    \vspace*{1cm}
  \end{center}
  
\end{folhadeaprovacao}
% ---

% ---
% Dedicatória
% ---
\begin{dedicatoria}
   \vspace*{\fill}
   \centering
   \noindent
   \textit{ Este trabalho é dedicado às crianças adultas que,\\
   quando pequenas, sonharam em se tornar cientistas.} \vspace*{\fill}
\end{dedicatoria}
% ---

% ---
% Agradecimentos
% ---
\begin{agradecimentos}
Os agradecimentos principais são direcionados à Gerald Weber, Miguel Frasson,
Leslie H. Watter, Bruno Parente Lima, Flávio de Vasconcellos Corrêa, Otavio Real
Salvador, Renato Machnievscz\footnote{Os nomes dos integrantes do primeiro
projeto abn\TeX\ foram extraídos de
\url{http://codigolivre.org.br/projects/abntex/}} e todos aqueles que
contribuíram para que a produção de trabalhos acadêmicos conforme
as normas ABNT com \LaTeX\ fosse possível.

Agradecimentos especiais são direcionados ao Centro de Pesquisa em Arquitetura
da Informação\footnote{\url{http://www.cpai.unb.br/}} da Universidade de
Brasília (CPAI), ao grupo de usuários
\emph{latex-br}\footnote{\url{http://groups.google.com/group/latex-br}} e aos
novos voluntários do grupo
\emph{\abnTeX}\footnote{\url{http://groups.google.com/group/abntex2} e
\url{http://www.abntex.net.br/}}~que contribuíram e que ainda
contribuirão para a evolução do \abnTeX.

\end{agradecimentos}
% ---

% ---
% Epígrafe
% ---
\begin{epigrafe}
    \vspace*{\fill}
	\begin{flushright}
        \epigraph{\em ``A day can really slip by when you're deliberately avoiding what you're
        supposed to do.''}
        {\em (Bill Watterson)}
	\end{flushright}
\end{epigrafe}
% ---

% ---
% RESUMOS
% ---

\setlength{\absparsep}{18pt} % ajusta o espaçamento dos parágrafos do resumo
\begin{resumo}
    The simulation of gas turbine engines is of vital importance to their design process and to the design of their control systems.
    In order to capture behaviour not shown by a simple thermodynamic cycle model, such as transients, unchoked turbine inlet, and instabilities, 
    as well as to increase the model fidelity, it is necessary to develop a component based engine model. 
    This work develops such engine model  for the simple case of a single spool turbojet, 
    with a single stage centrifugal compressor and single stage axial turbine. This configuration is typical of very small gas turbines.
    No experimental data is needeb as input to the model. It is based on first principles and semi-empirical corrections available in the literature.
    This work contains nonlinear models for a centrifugal compressor, axial turbine, nozzle, and the engine operating statically and dynamically.
    Surge is considered. The models are applied for the Jet Munts VT-80 engine and the results are compared with experimental data.

\end{resumo}

\begin{resumo}[Resumo]
 \begin{otherlanguage*}{brazil}
     A simulação de turbinas a gás é essencial para seu projeto e para o projeto dos seus sistemas de controle .
     Para simular características deste tipo de motor que não são capturadas por um simples modelo de ciclo termodinâmico, 
     e também para aumentar a fidelidade da simulação, é necessário desenvolver um modelo de componentes.
     Este trabalho desenvolve um modelo desse tipo para o caso de um motor turbojato de um único eixo, 
     com um compressor centrífugo de um estágio e uma turbina axial também de um único estágio.
     Essa configuração é típica para mini-turbinas a gás.
     Nenhum dado experimental é necessário como entrada para o modelo, pois ele é baseado em leis de conservação e correções semi-empíricas encontradas na literatura.
     Este trabalho contém modelos não lineares para compressores centrifugos, turbinas axiais, bocais, e motores completos operando em regime permanente ou transiente.
     A purga do compressor é levada em conta. Esses modelos são aplicados para o caso do motor Jet Munts VT-80 e os resultados comparados com dados experimentais.
     
   \noindent 
 \end{otherlanguage*}
\end{resumo}

% ---


% ---
% inserir lista de ilustrações
% ---
\pdfbookmark[0]{\listfigurename}{lof}
\listoffigures*
\cleardoublepage
% ---

% ---
% inserir lista de tabelas
% ---
\pdfbookmark[0]{\listtablename}{lot}
\listoftables*
\cleardoublepage
% ---

% ---
% inserir lista de abreviaturas e siglas
% ---
\begin{siglas}
  \item[ABNT] Associação Brasileira de Normas Técnicas
  \item[abnTeX] ABsurdas Normas para TeX
\end{siglas}
% ---

% ---
% inserir lista de símbolos
% ---
\begin{simbolos}
  \item[$ \Gamma $] Letra grega Gama
  \item[$ \Lambda $] Lambda
  \item[$ \zeta $] Letra grega minúscula zeta
  \item[$ \in $] Pertence
\end{simbolos}
% ---

% ---
% inserir o sumario
% ---
\pdfbookmark[0]{\contentsname}{toc}
\tableofcontents*
\cleardoublepage
% ---

\end{document}